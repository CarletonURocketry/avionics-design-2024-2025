\section{Previous Systems}

The previous year's avionics systems are used to guide this years designs. This section will briefly cover the flaws oflast year's systems to provide background on the decisions made this year.

\subsection{Flight Computer \& Transmitter}

The flight computer unit designed in the 2023-2024 academic year was endearingly referred to as the "QNX Stack" due to
its on-board \gls{mcu} running the \gls{qnx} \gls{rtos}. The assembled version of the unit which was flown at \gls{sac}
is pictured in Figure \ref{fig:preflight-stack}.

\subsubsection{Hardware Design}

The most glaring issue with the hardware design from last year was the selection of the Raspberry Pi 4b as the MCU for the system. The two largest issues with the Raspberry Pi were it's high power consumption and large mechanical size. 

The high power consumption resulted in the system only having around 8 hours of battery life, even with \qty{4000}{\milli\ampere\hour}. Although this number is already well below the idea battery life it could have been even lower depending on the task being performed. This is because the power consumption if a Raspberry Pi (like most MCU's) is highly dependent on the task being performed. Were the Pi performing more demanding tasks the battery life could have gotten as low as 2 hours or less.

The large mechanical size of the Raspberry Pi was also a major issue. The Raspberry Pi is a relatively large single board computer with many unused ports and connectors. Most of the features on this board were not used nor beneficial for the telemetry system and unnecessarily increased the size of the system to a point where it was extremely difficult to fit in the nosecone of the rocket.



\subsection{RF Design}
Write about issues with radio board signal (antenna placement, board design, enclosure reflection)
\fxwarning{Needs to be added.}

\subsection{Enclosure Design}
The placement of the antennas on the \gls{qnx} stack was not optimal. As can be seen in Figure
\ref{fig:preflight-stack}, the signal radiation from the longer transmitter antenna is partially blocked by the metal
enclosure, which affected transmission distance (discovered in later testing). The enclosure's reflective aluminum
material also increased signal interference, affecting the radiated signal that was not blocked but radiated forwards.
The \gls{gps} antenna (bright orange) also experienced blockage by the enclosure, resulting in a longer time to achieve
a fix.

Such a design impeded the ability of the system to transmit telemetry data and receive satellite \gls{gps} positioning
data. In particular, this would drastically reduce the signal quality received by the downlink ground station during
180 degrees of the rocket's roll, since the unobstructed portion of the antenna's radiation pattern would be facing
away from the receiver during that time.

\begin{figure}[H]
    \center
    \includegraphics[width=3in]{assets/images/stack.jpg}
    \caption{The 2023-2024 avionics flight computer unit, "QNX Stack"}
    \label{fig:preflight-stack}
\end{figure}

The QNX stack enclosure was sized in order to fit the largest member: the Raspberry Pi 4B mounted on a \gls{srad}
\gls{mcu} board. \Glspl{pcb} were sized to 90mm by 90mm, which the enclosure accommodated. This large size was
extremely tightly toleranced with the aluminum mounting bulkhead in order to fit within the rocket diameter. Figure
\ref{fig:stack-bulkhead} shows the mounting bulkhead around the enclosure.

Although initial fit-testing with \gls{cad} models was successful, the \gls{cad} model of the \gls{qnx} stack did not
include the antennas, or externally attached wiring with connectors for the electrical arming system. Figure
\ref{fig:stack-bulkhead} shows one of these connectors for supplying power to the system as it comes in contact with
the mounting bulkhead at the location where the \gls{qnx} stack was to be mounted using screws.

This connector, and the reasonably stiff wires it connects to, exceeded the clearance between the stack and bulkhead.
When the stack was inserted into the rocket nosecone, it required a series of twists and some gentle (perhaps not quite so gentle) pressure in order to clear the bulkhead so the arming wiring connectors and antennas slid between the threaded hole locations on the bulkhead. In fact, because this was discovered so close to the \gls{sac} flight, the final system required the connector in Figure \ref{fig:stack-bulkhead} to be bent downward to clear the bulkhead. Although this worked, it is not favourable to place mechanical stress on an electrical connector, much less one responsible for arming the entire telemetry system.

\begin{figure}[H]
    \center
    \includegraphics[width=3in]{assets/images/rad-cad.png}
    \caption{A CAD model of the radio board measured against the mounting bulkhead}
    \label{fig:rad-cad}
\end{figure}

The bulkhead clearance problem is also the reason for the close antenna placement shown in \Cref{fig:preflight-stack}. Any further spacing between the antennas would prevent them from clearing the bulkhead due to the protruding mounting holes on either side. A \gls{cad} visualization of this problem can be seen in \Cref{fig:rad-cad}. In fact, 90 degree \gls{sma} connector adapters were used on the final system because the 90 degree bend on the antennas themselves extended too far, causing them to hit the bulkhead edge when the stack was lowered into the nosecone.

\begin{figure}[H]
    \center
    \includegraphics[width=2.5in]{assets/images/stack-bulkhead.jpg}
    \caption{The enclosure within the aluminum bulkhead ring to which it was mounted}
    \label{fig:stack-bulkhead}
\end{figure}

The enclosure of the QNX stack exhibited buckling when it was recovered post-flight. The bent shape required that the supporting bulkhead be shaved down with a dremel tool in order to extract it. The buckling is visible \Cref{fig:stack-bent}. All boards survived despite the buckling pushing apart the 3D-printed rails far enough for the boards to come loose. No simulation of the enclosure was performed in advance of flight, which may have predicted these structural problems under shock load from launch and parachute deployment.

\begin{figure}[H]
    \center
    \includegraphics[width=3in]{assets/images/bent-stack.jpg}
    \caption{The recovered QNX stack post-flight, exhibiting buckling}
    \label{fig:stack-bent}
\end{figure}

\subsubsection{Software Design}

From a software perspective, last year's telemetry transmitter was very minimal in the quality of data that were transmitted.

Transmitted packets contained data read directly from the onboard sensors, performing no additional sensor fusion to measure different states of the rocket outside of using the altitude calculation suggested by the barometric pressure sensor data sheet for combining temperature and pressure. This resulted in noisy measurements about rocket state, like current altitude and angular velocity.

In addition, the system contained three sensors capable of measuring temperature. These sensors had a very high measurement frequency, and were being read as fast as possible using a 1.8GHz clock speed on a Raspberry Pi 4. This resulted in an overwhelming amount of temperature data being logged compared to other more important measurements such as altitude, acceleration and \gls{gps} position. This also meant much of the data being transmitted over radio was temperature data. The system design guaranteed that \gls{gps} data would always be sent when obtained because the \gls{gps} data was measured at only 10Hz, but this guarantee was never extended to other types of data.

The system was not robust in the way it handled failures. If the system had trouble detecting the \gls{eeprom} on the \gls{i2c} bus with startup configuration parameters, it would exit with a failure immediately instead of re-attempting to connect or provide an external indication to the operator that something went wrong with the electrical connection (no LED indication or buzzer, etc). This lack of robustness also applied to the radio process, which would fail if it could not connected to the radio module over \gls{uart}.

The system performed logging at its maximum frequency as soon as it was armed, which caused a significant amount of memory to be used for logging while the rocket was sitting idle on the launch pad. Although memory was not scarce with last year's system which used a large SD card, it did increase power consumption during the rocket's idle period. It is also inconvenient to sift through logs taken while idle to get the actual flight data.

Finally, there was no indication of battery life ever implemented on this system. The driver for the current/voltage sensor on-board was not completed in time for launch, so it was not possible to receive battery life warnings over telemetry or through a visual indicator in the final system.
