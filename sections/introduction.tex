\section{Introduction}

The avionics sub-team is the group within \glsxtrfull{cuinspace} that is responsible for all of the electronics on the
rocket(s) flown at competition each year.

Historically, this has included a telemetry system comprised of both a transmitter on board the rocket and a receiver
on the ground, as well as managing power distribution and wiring to \glsxtrfull{cots} and \glsxtrfull{srad} recovery
systems.

In the 2024-2025 year, avionics has seen significant member growth in both student numbers and skill level. The
sub-team is now designing and manufacturing a customized telemetry system \glsxtrfull{mcu} board for within the rocket,
performing more involved radio system testing and upgrading software systems to leverage sensor fusion and
\glsxtrfull{rtos} features.

The newest addition to the avionics sub-team's responsibilities this year is a hybrid rocket motor control system.
\glsxtrshort{cuinspace}'s first attempt at flying a hybrid rocket motor requires extensive testing of the motor itself,
and competition regulations impose strict design requirements for electronics that control the ground system plumbing
to the hybrid motor. The ground systems must also be operable from a remote location to maintain safe distance between
the operators and the pressurized oxidizer tank, which is done over a long range network interface.
