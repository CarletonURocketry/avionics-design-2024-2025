% ACRONYMS %
\newacronym{ui}{UI}{User Interface}
\newacronym{cuinspace}{CU InSpace}{Carleton University InSpace}
\newacronym{rtos}{RTOS}{Real-Time Operating System}
\newacronym{lora}{LoRa}{Long Range}
\newacronym{srad}{SRAD}{Student Researched And Designed}
\newacronym{cots}{COTS}{Commercial Off-The-Shelf}
\newacronym{posix}{POSIX}{Portable Operating System Interfaced based on UNIX}
\newacronym{ipc}{IPC}{Inter-Process Communication}
\newacronym{i2c}{I2C}{Inter-Integrated Circuit}
\newacronym{uart}{UART}{Universal Asynchronous Receiver/Transmitter}
\newacronym{html}{HTML}{Hyper Text Markup Language}
\newacronym{pcb}{PCB}{Printed Circuit Board}
\newacronym{gps}{GPS}{Global Positioning System}
\newacronym{gpio}{GPIO}{General Purpose Input/Output}
\newacronym{eeprom}{EEPROM}{Electrically Erasable Programmable Read-Only Memory}
\newacronym{mcu}{MCU}{Microcontroller Unit}
\newacronym{udp}{UDP}{User Datagram Protocol}
\newacronym{lan}{LAN}{Local Area Network}
\newacronym{tcp}{TCP}{Transmission Control Protocol}

% GLOSSARY DEFINITIONS %
\newglossaryentry{unix}{
    name=Unix,
    description={An operating system invented at Bell Labs which inspired a family of operating systems and POSIX
            standards}
}
\newglossaryentry{posixgls}{
    name=POSIX,
    description={A set of standards by the IEEE which define compatible operating system interfaces}
}
\newglossaryentry{gnu}{
    name=GNU,
    description={GNU's Not Unix; a collection of free software which can be used standalone or as an operating system}
}
\newglossaryentry{ham-radio}{
    name=HAM radio,
    description={A term for amateur radio, derived from the informal name for an amateur radio operator.}
}
