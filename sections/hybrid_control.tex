\section{Hybrid Control System}

This year's hybrid control system is a completely new project for avionics with no prior system to inform its design.
The current working prototype is an Arduino based control system which has been difficult to maintain and use. The
system was designed using breadboard electronics and it is very difficult to organize the jumper wires or perform
repairs on the system. It also does not have the ability to detect connection interruptions, the user interface does not
update at the speed desired for monitoring the system and it has a tendency to crash spuriously. This prototype system
has sufficed for preliminary testing of the hybrid in cold flows, but must be updated for static fire testing and
launch.

\subsection{Network Infrastructure}

Given that the launch pad and control base are separated by several hundred meters, it will be infeasible to establish 
a wired connection between the two. For this reason, communication between the launch pad and the control base will be
carried out over wireless ethernet using two Ubiquiti Litebeam M5 wireless network bridges. These network bridges will connect 
the \glsxtrfull{lan} at the control base and the \glsxtrshort{lan} at the launch pad, allowing them to communicate as if they 
were in the same LAN.

Each Litebeam M5 is powered using a 24V, 0.2A power over ethernet (PoE) adapter. <talk about specs>. 

Two types of communication will be handled over the network. The first is telemetry data sent from the launch pad which will be 
received by a telemetry client at the control base. Telemetry data will be transmitted from the launch pad to a \glsxtrfull{udp}
multicast group which the telemetry client will join. While transmitting data to a \glsxtrshort{udp} multicast address may be 
slightly slower than traditional \glsxtrshort{udp} unicast communication which transmits to a single device, transmitting to a 
multicast address provides redundancy in the sense that multiple clients are able to receive the same data should any single client
fail.

The second type of communication to be transmitted is control packets

\subsection{Control System Nodes}

\subsection{Hardware Design}
