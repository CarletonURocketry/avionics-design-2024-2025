\section{Hybrid Control System}

This year's hybrid control system is a completely new project for avionics with no prior system to inform its design.
The current working prototype is an Arduino based control system which has been difficult to maintain and use. The
system was designed using breadboard electronics and it is very difficult to organize the jumper wires or perform
repairs on the system. It also does not have the ability to detect connection interruptions, the user interface does not
update at the speed desired for monitoring the system and it has a tendency to crash spuriously. This prototype system
has sufficed for preliminary testing of the hybrid in cold flows, but must be updated for static fire testing and
launch.

\subsection{Network Infrastructure}

\subsection{Control System Nodes}

The communication within the hybrid control system network is centered around three distinct node types:

\begin{itemize}
    \item Telemetry clients
    \item Control client
    \item Pad control server
\end{itemize}

\subsubsection{Telemetry Clients}

Telemetry clients are simply consumers of the telemetry data produced by the pad control server. There can be a
theoretically infinite number of telemetry clients on the network at once, and they can all consume the same data being
sent by the pad control server. This is possible through the use of UDP multicast.

Telemetry clients are free to do whatever they would like with the telemetry data they receive. This could be simply
logging it to a file, performing a statistical analysis or using it to update status LEDs/hardware on a physical device.
The hybrid control UI is an example of a telemetry client which displays the data it receives visually.

\subsubsection{Control Client}

A control client is able to issue commands to the pad server. Only one control client is permitted to be connected to
the pad server at once. This is done to prevent the receipt of conflicting commands from two controllers, eliminating a
large suite of possible race conditions.

\subsection{Hardware Design}
